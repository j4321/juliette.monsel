%% start of file `template.tex'.
%% Copyright 2006-2013 Xavier Danaux (xdanaux@gmail.com).
%
% This work may be distributed and/or modified under the
% conditions of the LaTeX Project Public License version 1.3c,
% available at http://www.latex-project.org/lppl/.


\documentclass[12pt,a4paper,sans]{moderncv}        % possible options include font size ('10pt', '11pt' and '12pt'), paper size ('a4paper', 'letterpaper', 'a5paper', 'legalpaper', 'executivepaper' and 'landscape') and font family ('sans' and 'roman')

% moderncv themes
\moderncvstyle{perso}                             % style options are 'casual' (default), 'classic', 'oldstyle' and 'banking'
%\moderncvstyle{perso}                             % style options are 'casual' (default), 'classic', 'oldstyle' and 'banking'
\moderncvcolor{darkblue}                               % color options 'blue' (default), 'orange', 'green', 'red', 'purple', 'grey' and 'black'
%\renewcommand{\familydefault}{\sfdefault}         % to set the default font; use '\sfdefault' for the default sans serif font, '\rmdefault' for the default roman one, or any tex font name
%\nopagenumbers{}                                  % uncomment to suppress automatic page numbering for CVs longer than one page

% character encoding
\usepackage[utf8]{inputenc}                       % if you are not using xelatex ou lualatex, replace by the encoding you are using
\usepackage[english]{babel}
\usepackage[T1]{fontenc}
%\usepackage{multibib}
%\newcites{article,book}{{Articles},{Books}}
%\usepackage{CJKutf8}                              % if you need to use CJK to typeset your resume in Chinese, Japanese or Korean

% adjust the page margins
\usepackage[top=2cm, bottom=2cm, left=2cm, right=2cm]{geometry}

\setlength{\hintscolumnwidth}{3.2cm}                % if you want to change the width of the column with the dates
%\setlength{\makecvtitlenamewidth}{10cm}           % for the 'classic' style, if you want to force the width allocated to your name and avoid line breaks. be careful though, the length is normally calculated to avoid any overlap with your personal info; use this at your own typographical risks...
\renewcommand\colorhref[3][black]{\href{#2}{\textit{\small\color{#1}#3}}}

%%%%%%%%%%%%%%%%%%%%% publi and conf with biblatex %%%%%%%%%%%%%%%%%%%%%%%%%%%%%%%
\usepackage{csquotes}
\usepackage[backend=biber,
bibstyle=ieee,
citestyle=numeric,
sorting=ydmdddnt, 
date=year,
isbn=false,
url=true,
doi=false,
dashed=false,
maxnames=100]{biblatex}

\BiblatexSplitbibDefernumbersWarningOff
\AtEveryBibitem{%
    \clearfield{number}
}

\usepackage[normalem]{ulem} %for \uline

\usepackage{xstring}
\usepackage{etoolbox}
\newboolean{bold}


\DeclareSourcemap{
    \maps[datatype=bibtex]{
        \map{
            \step[fieldsource=dates]
            \step[fieldset=usera,origfieldval]
        }
    }
}


\DeclareSourcemap{
    \maps[datatype=bibtex]{
        \map{
            \step[fieldsource=video]
            \step[fieldset=userb,origfieldval]
        }
    }
}

\newcommand{\underlineauthor}[1]{%
    \DeclareNameFormat{author}{%
        \setboolean{bold}{false}%
        \renewcommand{\do}[1]{\expandafter\ifstrequal\expandafter{\namepartfamily}{####1}{\setboolean{bold}{true}}{}}%
        \docsvlist{#1}%
        \ifthenelse{\value{listcount}=1}
        {%
            {\expandafter\ifthenelse{\boolean{bold}}{\uline{\namepartgiveni\addspace\namepartfamily }}{\namepartgiveni\addspace\namepartfamily}}%
        }{\ifnumless{\value{listcount}}{\value{liststop}}
            {\expandafter\ifthenelse{\boolean{bold}}{\addcomma\addspace \uline{\namepartgiveni\addspace\namepartfamily }}{\addcomma\addspace \namepartgiveni\addspace\namepartfamily }}%
            {\expandafter\ifthenelse{\boolean{bold}}{\addcomma\addspace \uline{\namepartgiveni\addspace \namepartfamily \addcomma\isdot}}{\addcomma\addspace \namepartgiveni\addspace\namepartfamily\addcomma \addcomma\isdot}}%
        }
        \ifthenelse{\value{listcount}<\value{liststop}}
        {\addcomma\space}{}
    }
}
\ExecuteBibliographyOptions{doi=false}
\newbibmacro{string+doi}[1]{%
    \iffieldundef{doi}{#1}{\href{http://dx.doi.org/\thefield{doi}}{#1}}
}
\newbibmacro{string+url}[1]{%
    \iffieldundef{url}{#1}{\href{\thefield{url}}{#1}}
}

\newbibmacro{url+video}{%
    \iffieldundef{userb}{\iffieldundef{url}{}{Website: \url{\thefield{url}}}}{Video: \url{\thefield{userb}}}
}

\newbibmacro{dates}{%
    \iffieldundef{usera}{}{, \thefield{usera}}
}


\DeclareFieldFormat{title}{\usebibmacro{string+doi}{\mkbibemph{#1}}}
\DeclareFieldFormat{journaltitle}{\usebibmacro{string+doi}{\mkbibemph{#1}}}
\DeclareFieldFormat[inproceedings]{booktitle}{\usebibmacro{string+url}{\mkbibemph{#1}}}
\DeclareFieldFormat[inproceedings]{url}{\usebibmacro{url+video}}
%\DeclareListFormat[inproceedings]{location}{%
%        \usebibmacro{list:delim}{#1}%
%        #1\isdot
%        \usebibmacro{list:andothers}
%}

\underlineauthor{Monsel, Juliette}

\renewbibmacro*{date}{\ifentrytype{inproceedings}{\iffieldundef{usera}{}{, \thefield{usera}}}{}
}
%\renewbibmacro*{author/translator+others}{}
\DeclareFieldFormat[article]{pages}{\usebibmacro{string+doi}{#1}}
\DeclareFieldFormat[article]{volume}{\usebibmacro{string+doi}{\textbf{#1}}}
\renewbibmacro*{date+extrayear}{}
\renewbibmacro*{issue+date}{}

%
\newcommand*{\bibyear}{}
\renewbibmacro{in:}{}
\setlength{\bibitemsep}{0.2em}
\defbibenvironment{bibliography}
{\list
    {\iffieldequals{year}{\bibyear}
        {}
        {\printfield{year}%
            \savefield{year}{\bibyear}}}
    {\setlength{\topsep}{0pt}% layout parameters based on moderncvstyleclassic.sty
        \setlength{\labelwidth}{\hintscolumnwidth}%
        \setlength{\labelsep}{\separatorcolumnwidth}%
        \setlength{\itemsep}{\bibitemsep}%
        \leftmargin\labelwidth%
        \advance\leftmargin\labelsep}%
    \sloppy\clubpenalty4000\widowpenalty4000}
{\endlist}
{\item}

%\DeclareFieldFormat{usera}{\textbf{#1}}

%\AtEveryBibitem{%
%    \csappto{blx@bbx@\thefield{entrytype}}{% put at end of entry
%        \iffieldundef{usera}{}{%
%            \space\printfield{usera}
%        }
%    }
%}

\addbibresource{../publications.bib}
\addbibresource{../conf.bib}
\defbibheading{bibliography}[\refname]{}



\DeclareSortingTemplate{ydmdddnt}{
    \sort{
        \field{presort}
    }
    \sort[final]{
        \field{sortkey}
    }
    \sort[direction=descending]{
        \field{year}
    }
    \sort[direction=descending]{
        \field[padside=left,padwidth=2,padchar=0]{month}
        \literal{99}
    }
    \sort[direction=descending]{
        \field{day}
    }
    \sort{
        \field{journaltitle}
    }
    \sort{
        \field{author}
        \field{editor}
    }
    \sort{
        \field{title}
    }
} 
%%%%%%%%%%%%%%%%%%%%%%%%%%%%%%%%%%%%%%%%%%%%%%%%%%%%%%%%%%%%%%%%%%%%%%%%%%%%%%%%%%

%%%% add icons for resaerch social media
\usepackage{fontawesome5}
\renewcommand*\orcidsocialsymbol{{\faIcon{orcid}}~} % <=========================
%\newcommand*\researcheridsocialsymbol{{\faIcon{parking}}~} % <========= \aiGoogleScholarSquare
%\newcommand*\researchgatesocialsymbol{{\aiResearchGate}~}   % <========= \aiResearchGateSquare


% personal data
\name{Juliette}{Monsel}
%\title{Researcher in theoretical physics} 
\title{Researcher in theoretical physics}                               % optional, remove / comment the line if not wanted
\address{Gothenburg, Sweden}{}{}% optional, remove / comment the line if not wanted; the "postcode city" and "country" arguments can be omitted or provided empty
%\address{19 rue Revol}{38000 Grenoble}{France}% optional, remove / comment the line if not wanted; the "postcode city" and "country" arguments can be omitted or provided empty
%\phone[mobile]{+33 6 01 71 18 92}                   % optional, remove / comment the line if not wanted; the optional "type" of the phone can be "mobile" (default), "fixed" or "fax"
%\phone[fixed]{+2~(345)~678~901}
%\phone[fax]{+3~(456)~789~012}
\email{monsel@chalmers.se}                               % optional, remove / comment the line if not wanted
%\email{juliette.monsel@gmx.fr}                               % optional, remove / comment the line if not wanted
\homepage{j4321.github.io/juliette.monsel}                         % optional, remove / comment the line if not wanted
%\social[linkedin]{www.linkedin.com/in/juliette-monsel}                        % optional, remove / comment the line if not wanted
\social[orcid]{0000-0002-4965-6794}                        % optional, remove / comment the line if not wanted
\social[researcherid]{AAH-5591-2020} 

%\social[twitter]{jdoe}                             % optional, remove / comment the line if not wanted
%\social[github]{j4321}                              % optional, remove / comment the line if not wanted
\extrainfo{Nationality: French}                 % optional, remove / comment the line if not wanted
%\photo[64pt][0.4pt]{picture}                       % optional, remove / comment the line if not wanted; '64pt' is the height the picture must be resized to, 0.4pt is the thickness of the frame around it (put it to 0pt for no frame) and 'picture' is the name of the picture file
\quote{\vskip-5mm Research interests: stochastic thermodynamics, quantum open systems, quantum optics, optomechanics and electronic transport.}                                 % optional, remove / comment the line if not wanted

% to show numerical labels in the bibliography (default is to show no labels); only useful if you make citations in your resume
%\makeatletter
%\renewcommand*{\bibliographyitemlabel}{\@biblabel{\arabic{enumiv}}}
%\makeatother
%\renewcommand*{\bibliographyitemlabel}{[\arabic{enumiv}]}% CONSIDER REPLACING THE ABOVE BY THIS

% bibliography with mutiple entries
%\newcites{book,misc}{{Books},{Others}}
%----------------------------------------------------------------------------------
%            content
%----------------------------------------------------------------------------------
%\renewcommand*{\cvdoubleitem}[5][.25em]{%
% \cvitem[#1]{#2}{%
%   \begin{minipage}[t]{\doubleitemmaincolumnwidth+0.25\hintscolumnwidth}#3\end{minipage}%
%   \hspace*{\separatorcolumnwidth}%
%   \begin{minipage}[t]{0.5\hintscolumnwidth}\raggedleft\hintstyle{#4}\end{minipage}%
%   \hspace*{\separatorcolumnwidth}%
%   \begin{minipage}[t]{\doubleitemmaincolumnwidth+0.25\hintscolumnwidth}#5\end{minipage}}}

\begin{document}
\renewcommand{\labelitemi}{$\bullet$}
%\begin{CJK*}{UTF8}{gbsn}                          % to typeset your resume in Chinese using CJK
%-----       resume       ---------------------------------------------------------
\makecvtitle
%\vspace{-0.25cm}
%\vspace{0.5cm}
\vspace{-0.55cm}
\section{Education}
\cventry[0.5em]{2020}{Qualification}{National Council of Universities (CNU)}{France}{}{Certificate allowing me to apply to ``Maître de Conférence'' positions, permanent Assistant Professor positions with an important teaching component, at French universities.}%{{Dissertation:} Quantum thermodynamics and optomechanics.\\{Supervisor:} Alexia Auffèves}

\cventry[0.3em]{2019}{Ph.D.}{Université Grenoble Alpes}{France}{}{Theoretical Physics.}%{{Dissertation:} Quantum thermodynamics and optomechanics.\\{Supervisor:} Alexia Auffèves}
    
\cventry[0.3em]{2016}{M.Sc.}{École Normale Supérieure de Lyon}{France}{}{Major: Physics, Mention: highest honors}  % arguments 3 to 6 can be left empty
\cventry[0.3em]{2014}{B.Sc.}{École Normale Supérieure de Lyon}{France}{}{Major: Physics, Mention: highest honors}
\cventry[0.3em]{2011 -- 2013}{Classe Préparatoire}{Lycée La Martinière Monplaisir}{Lyon, France}{}{Two-year intensive course preparing for the competitive entrance examinations to French leading institutions of higher education. Track: Mathematics-Physics.}
%\cventry{July 2011}{Baccalauréat -- High school diploma}{Lycée du Val de Saône}{Trévoux, France}{}{Scientific course, Mention: highest honors, GPA: 18.2/20 -- A$^+$}

\section{Research experience}
%TODO: detail projects
\cventry[0.5em]{2020 -- current}{Postdoctoral researcher}{Department of Microtechnology and Nanoscience, Chalmers University of Technology}{Gothenburg, Sweden}{}{
    Advisor: Janine Splettstoesser. Quantum thermodynamics.
    \begin{itemize}
        \item Studied thermodynamic of electronic transport
        \item Analyzed optomechanical cooling in a thermodynamic perspective
    \end{itemize}
} 
\cventry[0.5em]{2019 -- 2020\\(4 months)}{Postdoctoral researcher}{Institut Néel}{Grenoble, France}{}{Advisor: Alexia Auffèves. Quantum thermodynamics and optomechanics.
\begin{itemize}
    \item Explored the potential of carbon nanotubes for thermodynamic experiments
    \item Studied stochastic thermodynamics with Kerr resonators
\end{itemize}} 
\cventry[0.5em]{2016 -- 2019\\(3 years, 2 months)}{Doctoral researcher}{Institut Néel}{Grenoble, France}{}
{
    Supervisor: Alexia Auffèves. Quantum thermodynamics and optomechanics.
    \begin{itemize}
        \item Demonstrated the potential of hybrid optomechanical systems and one-dimensional atoms to experimentally explore quantum thermodynamics
        \item Proposed methods to define and measure work in the quantum regime
    \end{itemize}
} 
\cventry[0.5em]{2016\\(4 months)}{Master intern}{Institut Néel}{Grenoble, France}{}{Supervisor: Alexia Auffèves. Fluctuation theorems in a hybrid optomechanical system.%
%\begin{itemize}%
%\item Simulated trajectories of the hybrid optomechanical system using a quantum jump approach
%\item Computed the irreversibility for a single trajectory and checked Jarzynski's equality
%\end{itemize}
}
\cventry[0.5em]{2015\\(3 months)}{Master intern}{Institut Néel}{Grenoble, France}{}{Supervisor: Alexia Auffèves. Hybrid optomechanical system in the ultra-strong coupling regime.%
%\begin{itemize}%
%    \item Simulated trajectories of the hybrid optomechanical system using Python programming language
%    \item Showed that the system can behave like a ``mechanical laser'' or a cooling system
%\end{itemize} 
}
%%\cventry{2014\\(6 Fridays)}{Experimental Project}{Laboratoire de Physique}{ENS de Lyon}{supervisor: Antoine Naert}{Energy exchanges with a dissipative thermostat.
%\begin{itemize}
%\item Studied the energy exchanges between a rarefacted granular gas and a blade
%\end{itemize}
%}
\cventry[0.5em]{2014\\(2 months)}{Bachelor intern}{Institut Lumière Matière}{Lyon, France}{}{Supervisor: Julien Laverdant. Experimental control of polarization with a spatial light modulator.%
%%\begin{itemize}%
%%\item Simulated the focusing of a laser beam with Matlab
%%\item Successfully generated a radially polarized beam with the spatial light modulator
%%\end{itemize}
}
%\newpage
\section{Teaching experience}
\cventry{2017, 2018\\(64 hours/year)}{Teaching Assistant}{Université Grenoble Alpes}{France}{}{Newtonian mechanics for first year undergraduates.
\begin{itemize}
    \item Supervised students during tutorials (2$\times$1,5 hours/week, $\sim$ 30 students) and practical work (3 hours/week, $\sim$ 15 students)
    \item Graded examinations and practical work reports
    \item Wrote exercises for the examinations
\end{itemize}
}

\cventry{2013 -- 2014\\(7 months)}{Tutor for homework assistance}{Trait d'Union program}{Villeurbanne, France}{}{Took part in a homework assistance program for students from high schools in disadvantaged areas (2 hours/week).
    %    \begin{itemize}
    %        \item Supervised high school students while they were doing their homework (two hours a week)
    %        \item Met people from varied social and cultural background
    %    \end{itemize}
}
\subsection{Student supervision}
\cvitem[0.5em]{Oct. 2021 -- current}{Co-supervisor of the master student Juliane Graf from University of Regensburg (Germany) at MC2, Chalmers University of Technology for a six-month project.}
\cvitem[0.5em]{2020 -- current}{Official assistant supervisor of the Ph.D. student Ludovico Tesser at MC2, Chalmers University of Technology.}
\cvitem[0.5em]{2016}{Informal co-supervisor of the master student Cauê Moreno Kersul de Castro Carvalho during his two-month internship at the Institut Néel, Grenoble, France.}

\subsection{Training}
\cventry[0.5em]{2017}{How to develop as a teacher}{Doctoral school of Physics}{Grenoble, France}{}{Two-day training program on communication and group animation techniques for teaching.}
\cventry[0.5em]{2016 -- 2019}{Research and Higher Education (RES) label}{Doctoral school of Physics}{Grenoble, France}{}{Teaching oriented Ph.D. program leading to the production of a portfolio documenting the development of my teaching and research skills (in French): \url{http://juliette-monsel.byethost15.com}.}
\cventry[0.5em]{2016}{Introduction to the profession of teacher-researcher}{Doctoral school of Physics}{Autrans, France}{}{Three-day workshop on topics related to teaching at the university.}
%\vspace{0.2cm}
\section{Awards and Grants}
\cvitem{Submitted}{HORIZON Research and Innovation Actions: Basic Science for Quantum Technologies, as part of a European consortium.}
\cvitem{2020}{Spinger Thesis Award, recognizing outstanding Ph.D. research.}
\cvitem{2016}{Ph.D. grant from the CFM Foundation for Research.}
%\vspace{0.2cm}


%\vspace{0.2cm}
\nocite{*}
\section{Publications}
\subsection{Articles}
\printbibliography[title={Articles}, keyword={publi}]
\renewcommand*{\bibyear}{}
\subsection{Books}
\printbibliography[title={Books}, keyword={book}]
\renewcommand*{\bibyear}{}

%\nocitebook{book1,book2}
%\bibliographystylebook{plain}
%
%\bibliography{publications_academic}                   % 'publications\documentclass[options]{class}' is the name of a BibTeX file
%\bibliographystyle{moderncv}

%\vspace{0.2cm}
\section{Conferences and seminars} 
%TODO: Add details and month
\subsection{Invited talks}
\printbibliography[title={Invited talks}, keyword={invited}]
\renewcommand*{\bibyear}{}
\vspace{0.1cm}
\subsection{Contributed talks}
\printbibliography[title={Contributed talks}, keyword={contributed}]
\renewcommand*{\bibyear}{}
\vspace{0.1cm}
\subsection{Invited seminars}
\printbibliography[title={Invited seminars}, keyword={seminar}]
\renewcommand*{\bibyear}{}
\vspace{0.1cm}
\subsection{Contributed posters}
\printbibliography[title={Contributed posters}, keyword={poster}]
\renewcommand*{\bibyear}{}


%\section{Supervision experience}
%... PhD co-supervisor ...

\section{Service to the community}
\cvitem[0.5em]{Reviewer}{Phys. Rev. A (2021), J. Stat. Mech. Theory Exp. (2021), Phys. Rev. E (2021), J. Phys. A Math. (2021), New J. Phys. (2020), Commun. Phys. (2020)}
\cvitem[0.5em]{Scientific outreach}{Speaker and guide (2016 -- 2019) at the ``Fête de la Science'', a yearly national French event during which scientific institutions promote science through animations and laboratory tours aimed at high school students and the general public.}

\section{Volunteer experience}
\cventry[0.5em]{2020 -- current}{Cykelköket}{}{Gothenburg, Sweden}{}{The ``Bike kitchen'' is an open Do-It-Yourself bicycle workshop.
    \begin{itemize}
        \item Helped people repair their bikes
        \item Took part in the administration of the workshop as a board member
    \end{itemize}
}
\cventry[0.5em]{2017 -- 2020}{uN p’Tit véLo dAnS La Tête}{}{Grenoble, France}{}{Associative self-repair workshop aiming at teaching bicycle mechanics and promoting bike riding.
    \begin{itemize}
        \item Learned bicycle mechanics by dismantling and repairing bikes for the association
        \item Explained to members of the association how to repair their bikes
        \item Took part in meetings and helped organize events as a member of the board from September 2018 to February 2020
    \end{itemize}
}
%\cventry{2013 -- 2014\\(7 months)}{Tutor for homework assistance}{Trait d'Union program}{Villeurbanne, France}{}{Homework assistance program for students from high schools in disadvantaged areas (2 hours/week).
    %    \begin{itemize}
    %        \item Supervised high school students while they were doing their homework (two hours a week)
    %        \item Met people from varied social and cultural background
    %    \end{itemize}
%}

\vspace{0.2cm}
\section{Skills}
\subsection{Languages}
%\cvitemwithcomment[0.4em]{English}{fluent}{CLES 2 certificate of B2 level}
%\cvitem[0.4em]{English}{fluent}
%\cvitem[0.4em]{French}{native speaker}
%\cvitem[0.4em]{Italian}{good oral and written comprehension}
%%\cvitemwithcomment[0.4em]{Swedish}{currently learning}{following B1 course at Chalmers}
%\cvitem[0.4em]{Swedish}{currently learning}
\cvdoubleitem[0.4em]{English}{fluent}{Italian}{good comprehension}
\cvdoubleitem[0.4em]{French}{native speaker}{Swedish}{currently learning}
%\cvitem{Languages}{\textbf{French}: native\quad \textbf{English}: fluent\quad \textbf{Italian}: average}
%\cvitem{Computer}{\textbf{Programming}: Python, \LaTeX\quad \textbf{Softwares}: Matlab, Spyder\quad \textbf{OS}: Windows, Linux}
\vspace{0.2cm}
\subsection{Computer}
\cvitem[0.4em]{Programming}{Python, Git, Matlab, C++}
\cvitem[0.4em]{Operating systems}{Linux, Windows, MacOS}
\cvitem[0.4em]{Text processing}{\LaTeX, LibreOffice}
%\cvitem{Computer}{\textbf{Programming}: Python, \LaTeX, Git\quad \textbf{Softwares}: Matlab, Spyder}
%\vspace{0.2cm}
\section{Interests}
%~\cvitem[0.5em]{Handiwork}{bicycle mechanics, crochet, sewing}
%\cvitem[0.5em]{Reading}{novels, mostly mysteries, in French (Fred Vargas) and in English (Michael Connelly, Peter Robinson)}
%\cvitem[0.5em]{Sports}{hiking, cycling}
%\cvitem[0.5em]{Programming}{\colorhref{https://github.com/j4321/}{open-source software} development, answering questions on \colorhref{https://stackoverflow.com/users/6415268/j-4321}{StackOverflow}}

\cvitem[0.4em]{Bicycle}{My main mean of transportation since 2017, volunteer mechanics and board member in Do-It-Yourself bicycle workshops.}
\cvitem[0.4em]{Programming}{\colorhref{https://github.com/j4321/}{Open-source software} development with Python, answering questions on \colorhref{https://stackoverflow.com/users/6415268/j-4321}{StackOverflow}.}
\cvitem[0.4em]{Reading}{Novels, mostly mysteries, in English (Michael Connelly, Peter Robinson) and in French (Fred Vargas).}
%\section{References}
%%\begin{minipage}[t]{0.5\maincolumnwidth+0.45\hintscolumnwidth}
%%    \textbf{Alexia Auffèves (Ph.D. supervisor)}\\
%%    Institut Néel, CNRS - Université Grenoble Alpes\\
%%    25 rue des Martyrs BP 166\\
%%    38042 Grenoble cedex 9, France\\
%%    +33 4 76 88 79 27 -- alexia.auffeves@neel.cnrs.fr
%%\end{minipage}
%%\begin{minipage}[t]{0.5\maincolumnwidth+0.65\hintscolumnwidth}
%%    \textbf{François Gieres (master teacher)}\\
%%    Institut de Physique des 2 Infinis, Université Claude Bernard\\
%%    4, rue Enrico Fermi\\
%%    69622 Villeurbanne Cedex, France\\
%%    +33 4 72 43 26 81 -- f.gieres@ipnl.in2p3.fr
%%\end{minipage}
%%%%
%%%%    
%\cventry{}{Janine Splettstoesser}{Postdoctoral advisor}{}{}{
%%    Chalmers University of Technology, SE-412 96 Gothenburg, Sweden.\\
%    +46 31 772 3111 -- janines@chalmers.se\\[0.2cm]
%    Chalmers University of Technology\\
%    SE-412 96 Gothenburg\\
%    Sweden
%}
%\cventry{}{Alexia Auffèves}{Ph.D. supervisor}{}{}{
%%    Univ. Grenoble Alpes, CNRS, Grenoble INP, Institut Néel, 38000 Grenoble, France.
%    +33 4 76 88 79 27 -- alexia.auffeves@neel.cnrs.fr\\[0.2cm]
%    Institut Néel, CNRS - Université Grenoble Alpes\\
%    25 rue des Martyrs BP 166\\
%    38042 Grenoble cedex 9\\
%    France
%}

\end{document}


